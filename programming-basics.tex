%! TeX root = main.tex

\chapter{Programming Basics}

The C++ code for the Arduino has the same basic elements as those of any programming language. 
The code consists of a series of \emph{statements} that are translated (compiled) into instructions for the processor to execute.
Statements can contain \emph{expressions}, which are logical or mathematical \emph{operations} performed on memory \emph{variables} or constants.
The \emph{flow} of the program, or which instructions are executed and whether they are repeated, can be controlled to implement complex functionality.

Finally, to reduce complexity and simplify program development, some abstractions can be introduced to better organize the code.
Code that is reused multiple times can be made into \emph{functions}, variables that exist together can be grouped into \emph{structures}, and \emph{classes} can be used to tie structures and functions together in hierarchical patterns.

In this chapter, we will go over the basic concepts highlighted above and how they are implemented in the C++ language, on the Arduino platform.

\section{Operators and Expressions}
\section{Variables and Scopes}
\section{Statements}
\section{Operators}
\section{Flow Control}
\section{Functions}
\section{Structures}
\section{Classes}


%%% Local Variables:
%%% TeX-master: "main"
%%% eval: (adaptive-wrap-prefix-mode t)
%%% eval: (visual-line-mode t)
%%% eval: (nlinum-mode t)
%%% TeX-engine: luatex
%%% End: