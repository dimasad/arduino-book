%! TeX root = main.tex

\chapter{Programming Basics}

The C++ code for the Arduino has the same basic elements as those of any programming language. 
The code consists of a series of \emph{statements} that are translated (compiled) into instructions for the processor to execute.
Statements can contain \emph{expressions}, which are logical or mathematical \emph{operations} performed on memory \emph{variables} or constants.
The \emph{flow} of the program, or which instructions are executed and whether they are repeated, can be controlled to implement complex functionality.

When working in teams and making large projects, the program source code can become hard to follow and understand.
to reduce complexity and simplify program development, some abstractions can be introduced to better organize the code.
Code that is reused multiple times can be made into \emph{functions}.
Variables that exist together can be grouped into \emph{structures} and \emph{classes} can be used to tie structures and functions together in hierarchical patterns.

In this chapter, we will go over the basic concepts highlighted above and how they are implemented in the C++ language, on the Arduino platform.

\section{Variables and Scopes}
Data in the microcontroller's memory is organized as \emph{variables} in C++ code.
In C and C++, each variable must be declared before it is used.
The declaration tells the compiler what is the variable type, its name, and optionally its initial value.

As an example, \inocodei{int x;} declares an integer variable named ``x''.
Variables are often declared with an initial value, like in the code \inocodei{float a = 1.5;} which declares a floating-point (number with decimal part) variable with initial value \num{1.5}.
Multiple variables of the same type can be declared together, like in the code \inocodei{int a,b,c;} which declared three variables, named ``a'', ``b'', and ``c'', of integer type.
Variable declarations, as other \emph{statements} in C++, must end in a semicolon.

Every variable's identifier is its name.
In C++ an identifier must obey four simple rules:

\begin{enumerate}
  \item 
  The first character is a letter or an underscore (\texttt{\_}).
  Digits are not allowed in the first position.
  \item 
  The remaining characters may be letters, digits or underscores.
  No spaces, hyphens or other symbols are allowed.
  \item 
  Identifiers are case-sensitive, so \inocodei{motor}, \inocodei{Motor} and \inocodei{MOTOR} can identify three different variables.
  \item 
  The identifier cannot match a reserved keyword such as \inocodei{int}, \inocodei{for}, or \inocodei{while}, for example.
\end{enumerate}

Examples of valid identifiers are \inocodei{button1_state}, \inocodei{_vel}, while \inocodei{1button}, {error!} and \inocodei{time-frame} cannot be used as identifiers.
To write good code, please choose informative variable identifiers.
This is a skill that is developed over time, but perfer identifiers that capture purpose: \inocodei{ambientTemperatureC} or \inocodei{motorSpeedRPM} say far more than generic placeholders like \inocodei{temp}, \inocodei{var}, or \inocodei{temporary}.
Descriptive names double as in-line documentation, speeding reviews and maintenance, and they appear at the top of virtually every modern C++ style guide.

In embedded work you will meet two main naming styles: \inocodei{snake_case}, which is used in the Arduino core, and \inocodei{camelCase}, used in much of modern C++. 
Pick one and stay consistent---it helps readability and coding in a team.

- Variable types and type modifiers.

- Scopes.
A variable cannot be redeclared on the same scope.

\section{Operators and Expressions}

\section{Statements}
\section{Operators}
\section{Flow Control}
\section{Functions}
\section{Structures}
\section{Classes}


%%% Local Variables:
%%% TeX-master: "main"
%%% eval: (adaptive-wrap-prefix-mode t)
%%% eval: (visual-line-mode t)
%%% eval: (nlinum-mode t)
%%% TeX-engine: luatex
%%% End: