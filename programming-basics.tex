%! TeX root = main.tex

\chapter{Programming Basics}

The C++ code for the Arduino has the same basic elements as those of any programming language. 
The code consists of a series of \emph{statements} that are translated (compiled) into instructions for the processor to execute.
Statements can contain \emph{expressions}, which are logical or mathematical \emph{operations} performed on memory \emph{variables} or constants.
The \emph{flow} of the program, or which instructions are executed and whether they are repeated, can be controlled to implement complex functionality.

When working in teams and making large projects, the program source code can become hard to follow and understand.
to reduce complexity and simplify program development, some abstractions can be introduced to better organize the code.
Code that is reused multiple times can be made into \emph{functions}.
Variables that exist together can be grouped into \emph{structures} and \emph{classes} can be used to tie structures and functions together in hierarchical patterns.

In this chapter, we will go over the basic concepts highlighted above and how they are implemented in the C and C++ languages, on the Arduino platform.

\section{Variables}
Data in the microcontroller's memory is organized as \keywd{variables} in C++ code.
In C and C++, each variable must be declared before it is used.
The declaration tells the compiler what is the variable type, its name, and optionally its initial value.

As an example, \inocodei{int x;} declares an integer variable named ``x''.
Variables are often declared with an initial value, like in the code \inocodei{float foo = 1.5;} which declares a floating-point (number with decimal part) variable named ``foo'' with initial value \num{1.5}.
Multiple variables of the same type can be declared together, like in the code \inocodei{int a,b,c;} which declared three variables, named ``a'', ``b'', and ``c'', of integer type.
Variable declarations, as other \keywd{statements} in C++, must end in a semicolon.

\subsection{Identifiers}
The name of a variable is formally called its \keywd{identifier} in C and C++.
An identifier in C++ must obey the following simple rules:

\begin{enumerate}
  \item 
  The first character is a letter or an underscore (\texttt{\_}).
  Digits are not allowed in the first position.
  \item 
  The remaining characters may be letters, digits or underscores.
  No spaces, hyphens or other symbols are allowed.
  \item 
  Identifiers are case-sensitive, so \inocodei{motor}, \inocodei{Motor} and \inocodei{MOTOR} identify three different variables.
  \item 
  The identifier cannot match a reserved keyword such as \inocodei{int}, \inocodei{for}, or \inocodei{while}, for example.
\end{enumerate}

Examples of valid identifiers are \inocodei{button1_state}, \inocodei{_vel}, while \inocodei{1button}, \inocodei{error!} and \inocodei{time-frame} cannot be used as identifiers.
To write good code, please choose informative variable identifiers.
This is a skill that is developed over time, but perfer identifiers that capture purpose: \inocodei{ambientTemperatureC} or \inocodei{motorSpeedRPM} say far more than generic placeholders like \inocodei{temp}, \inocodei{var}, or \inocodei{temporary}.
Descriptive names double as in-line documentation, speeding reviews and maintenance, and they appear at the top of virtually every modern programming style guide.

\subsection{Types}
The C and C++ languages are strongly typed, meaning that a variable is declared with a specific type that cannot change throughout its lifetime.
For example, if a variable is declared as an integer (\inocodei{int}) it cannot hold a value with a fractional part.
When a floating-point value is assigned to an integer variable, the fractional part is truncated before assignment---that is, the number is rounded toward zero.
The fundamental built-in types of C++ can be divided into four broad families:
\begin{description}
\item[Integers:] whole numbers without a fractional part. They come in signed and unsigned varieties and four standard widths.
\item[Floating-point numbers:] real numbers that can carry a fractional component.
\item[Boolean:] a logic variable that takes values true or false.
\item[Characters:] byte-size numbers that are designed for text but are equally handy for raw bytes, bit-masks, and small lookup tables.
\item[Pointers:] a number that represents a location in the processor's memory.
\item[Void:] the \emph{no-type} type. It represents the absence of a value, like in a function that does not return anything.
\end{description}

\begin{table}[b]
  \caption{%
    Fundamental types in C++ and their width in the AVR architecture   (Arduino Uno Rev3), ARM (Raspberry Pi Pico 2 or Arduino Due), and the minimum requirements of the C++ standard.
    The rightmost column gives the minumum range of values a variable of this type should be able to represent, according to the standard.
  }
  \label{tab:cpp-fundamental}
  \pgfplotstabletypeset[
      col sep=&, row sep=\\, string type,
      columns/Keyword/.style={column type={>{\ttfamily}r}},
      every head row/.style={
        output empty row,              % ← suppress automatic header
        before row={
          \toprule
          % first custom header line
          \multicolumn{1}{c}{} % empty cell over Keyword
          &\multicolumn{3}{c}{\emph{Number of bits}}  % merged over three cols
          &\multicolumn{1}{c}{} % empty cell
          \\
          \cmidrule(lr){2-4}
          % second header line (per-column labels)
          \textrm{\textbf{Keyword}} & AVR & ARM & Min. &
          \textbf{Min. Range} \\
          \midrule
        },
        after row={},
      }
    ]{%
    Keyword & BitsAVR & BitsARM & MinBits& Min. Range\\
    bool & 8 & 8 & --- & 0 (false) or 1 (true) \\
    signed char& 8 & 8 & 8 & \num{-128} to \num{127} \\
    unsigned char& 8 & 8 & 8 & \num{0} to \num{255} \\
    signed short int& 16 & 16 & 16 & \num{-32768} to \num{32767} \\
    unsigned short int& 16 & 16 & 16 & \num{0} to \num{65535} \\
    signed int& 16 & 32 & 16 & \num{-32768} to \num{32767} \\
    unsigned int& 16 & 32 & 16 & \num{0} to \num{65535} \\
    signed long int& 32 & 32 & 32 & \num{+-2.1e9} \\
    unsigned long int& 32 & 32 & 32 & \num{0} to \num{4.3e9} \\
    signed long long int& 64 & 64 & 64 & \num{+-9.2e18} \\
    unsigned long long int& 64 & 64 & 64 & \num{0} to \num{1.8e19} \\
    float & 32 & 32 & 32 & \num{+-3.4e38} \\
    double & 32 & 64 & 64 & \num{+-1.8e308} \\
  }  
\end{table}

A variable's type also determines how much memory is required to store it and, consequently, the range of values it can represent.
A summary of the size of some fundamental types and the minimum range they can represent is shown in Table~\ref{tab:cpp-fundamental}, along with their standard width in some processor architectures that are used with the Arduino platform.
The keywords \inocodei{short}, \inocodei{long}, and \inocodei{long long} are modifiers that ask the compiler for progressively wider integers, while \inocodei{signed} and \inocodei{unsigned} include or remove the sign bit which can half or double the positive range of a number. 
The C++ standard specifies only minimum widths, so the exact size depends on the tools and configuration used.
Note that not all compilers comply fully with the standard---the compiler of the AVR architecture for the Arduino Uno Rev3, for example, uses only 32 bits to store a \texttt{double}, less than the minimum 64 bits required by the standard.

The type specifiers shown in Table~\ref{tab:cpp-fundamental} are quite long, however.
The C++ language accepts shorthands which are often used in code.
The specifiers \inocodei{int} and \inocodei{signed} can be ommited when used with other specifiers.
A summary of common equivalent specifiers is shown in Table~\ref{tab:type-specifiers-shorthand}.

\begin{table}
  \centering
  \caption{Examples of equivalent type specifiers in C++.}
  \label{tab:type-specifiers-shorthand}
  \pgfplotstabletypeset[
    col sep=&, row sep=\\, string type,
  ]{%
    Common shorthand & Long form \\
    \texttt{int} & \texttt{signed int} \\
    \texttt{long} & \texttt{signed long int} \\
    \texttt{unsigned} & \texttt{unsigned int} \\
    \texttt{unsigned long int} & \texttt{unsigned long} \\
  }  
\end{table}

The standard C/C++ libraries also specify integers with an \emph{exact} number of bits to help write platform-independent code.
This is especially important when writing software libraries.
The specifiers \inocodei{int8_t}, \inocodei{int16_t}, \inocodei{int32_t}, and \inocodei{int64_t} stand for signed integers of exactly \num{8}, \num{16}, \num{32}, and \num{64} bits, respectively, while \inocodei{uint8_t}, \inocodei{uint16_t}, \inocodei{uint32_t}, and \inocodei{uint64_t} represent the equivalent unsigned integers.

% The paragraph below was written mostly by chatgpt.
% https://chatgpt.com/share/6862a592-7ab0-8012-8d13-9f8228d3c536
In addition to type and scope, C++ also allows you to further qualify variables with the keywords \inocodei{const} and \inocodei{constexpr}. 
The \inocodei{const} keyword marks a variable as read-only---it must be assigned once, and any further attempt to change its value will result in a compile-time error.
The \inocodei{constexpr} keyword goes a step further: it requires the value to be known at compile time. 
On microcontrollers, this means the variable will not occupy RAM, since its value is embedded directly into the program code stored in flash memory. 
This is especially important for values like pin numbers, array sizes, or fixed configuration constants in resource-constrained systems.

\subsection{Scopes}
When projects grow, there arises a need to reuse variable names in different contexts.
When authoring a function or a code fragment, development is also simplified if the new code added won't have unintended consequences elsewhere.
The concept of \keywd{scopes} helps us organize that.

At the top level, there is the \keywd{global scope}%
\footnote{%
  In the C standard, the global scope is officially called \keywd{file scope} while in the C++ standard it is the \keywd{global namespace scope}, but they are essentially the same. \cneed
}, which has variables that can be accessed throughout the whole program.
Whether to make a variable global or not is a programming choice
Simple, short programs often use global variables.
In larger codebases, however, global variables can make the code difficult to understand and debug because \emph{any} line of code can use or alter them.
To reduce the cognitive load on the programmer, it is helpful to restrict variables to smaller chunks of code called \keywd{blocks}.

Blocks are sections of code between curly brackets \texttt{\{\}}.
In the example code \texttt{Scopes.ino}, shown below, there are two blocks, the \texttt{setup} and \texttt{loop} functions.
Variables declared in a block scope exist in the computer memory during the execution of the lines of code between the variable declaration and the end of the block.
Variables declared outside any block are global.
In the \texttt{Scopes.ino} program, the variable \texttt{foo} is global while \texttt{bar} is local to the \texttt{loop} function block.
Note that the global variable \texttt{foo} can be accessed (written to and read) inside any block of code.

\inofile{Scopes.ino}

A variable cannot be declared more than once within a scope using the same identifier.
Different variables with the same identifier can be declared in different scopes, however.
When multiple variables with the same identifier are declared in different scopes, only the one in the innermost scope will be accessed.
In this case, we say that the variables in the outer scopes are \emph{shadowed} by the local one.

Blocks can be created inside blocks, like shown in the example program \texttt{NestedScopes.ino}.
A new block, inside the block of the \texttt{setup} function, starts at line 5 and ends at line 7.
The variable \texttt{foo} declared at line 6 shadows the global variable \texttt{foo} declared at line 1.
Note that the new variable can be of any type; in this example its type is different from the one in the outer scope.
Nested blocks are usually associated with control flow statements such as if/else, for, or while.

\inofile{NestedScopes.ino}

\section{Operators and Expressions}
The basic mathematical and logic operations on variables are performed with \keywd{operators} like addition, multiplication, division, and so on.
One or more operators form an expression that has yields a value that can be used for program flow control or be assigned to another variable.

Like in standard math notation, some operators have \emph{precedence} with respect to others.
Higher precedence operators, like multiplication, are evaluated before lower precedence ones, like addition.
Table~\ref{tab:operators} lists common operators in the C and C++ languages sorted by order of precedence.
A pair of matching parentheses \texttt{()} can be used to group expressions in order to change the order of evaluation%
\footnote{%
  In my years as an instructor, I've seen students add a baffling number of parentheses, around every operator, to make sure the expression is calculated in the order they want.
  They usually do this because they are unsure about the order of precedence and there is a bug---usually unrelated to the expressions being changed---whose root they cannot find.
  While this has no effect on the program behavior, it makes the code harder to read, so try to avoid it.
  When in doubt, a quick internet search for ``C/C++ operator precedence'' will give multiple results for reference.
}.
If two operators have the same precedence, like addition and subtraction, the expression is evaluated from left to right.

\begin{table}
  \centering
  \caption{Common operators in C and C++, groups from higher to lower precedence order.}
  \label{tab:operators}
  \small
  \begin{tabular}{cp{6cm}c}
    \textbf{Operator} & \textbf{Description} & \textbf{Example} \\
    \hline\hline
    \multicolumn{3}{c}{\emph{Precedence: primary}} \\
    \texttt{[]} & Array subscripting. & \texttt{a[i]} \\
    \texttt{()} & Function call.      & \texttt{f(x)} \\
    \texttt{.}  & Member access.      & \texttt{o.m}  \\
    \texttt{++} & Increment.     & \texttt{i++}  \\
    \texttt{--} & Decrement.     & \texttt{i--}  \\
    \hline
    \multicolumn{3}{c}{\emph{Precedence: unary}} \\
    \texttt{-}  & Unary minus. & \texttt{-i}  \\
    \texttt{!}  & Logical NOT. & \texttt{!a}  \\
    \texttt{\raisebox{0.5ex}{\texttildelow}}  & Bitwise NOT.     & \texttt{\raisebox{0.5ex}{\texttildelow}a}  \\
    \hline
    \multicolumn{3}{c}{\emph{Precedence: multiplicative}} \\
    \texttt{*}  & Multiplication.                  & \texttt{a * b} \\
    \texttt{/}  & Division.                        & \texttt{a / b} \\
    \texttt{\%} & Remainder of division (modulus). & \texttt{a \% b}\\
    \hline
    \multicolumn{3}{c}{\emph{Precedence: additive}} \\
    \texttt{+}             & Addition.                         & \texttt{a + b}       \\
    \texttt{-}             & Subtraction.                      & \texttt{a - b}       \\
    \hline
    \multicolumn{3}{c}{\emph{Precedence: relational}} \\
    \texttt{<}             & Less than.                        & \texttt{a < b}       \\
    \texttt{<=}            & Less than or equal.             & \texttt{a <= b}      \\
    \texttt{>}             & Greater than.                     & \texttt{a > b}       \\
    \texttt{>=}            & Greater than or equal.          & \texttt{a >= b}      \\
    \hline
    \multicolumn{3}{c}{\emph{Precedence: equality}} \\
    \texttt{==}            & Equality.                         & \texttt{a == b}      \\
    \texttt{!=}            & Inequality.                       & \texttt{a != b}      \\
    \hline
    \multicolumn{3}{c}{\emph{Precedence: bitwise AND}} \\
    \texttt{\&}            & Bitwise and.                      & \texttt{a \& b}      \\
    \hline
    \multicolumn{3}{c}{\emph{Precedence: bitwise XOR}} \\
    \texttt{\^}            & Bitwise exclusive or (XOR).                      & \texttt{a \^{ } b}      \\
    \hline
    \multicolumn{3}{c}{\emph{Precedence: bitwise OR}} \\
    \texttt{|}             & Bitwise or.                       & \texttt{a | b}       \\
    \hline
    \multicolumn{3}{c}{\emph{Precedence: logical AND}} \\
    \texttt{\&\&}          & Logical and.                      & \texttt{a \&\& b}    \\
    \hline
    \multicolumn{3}{c}{\emph{Precedence: logical OR}} \\
    \texttt{||}            & Logical or.                       & \texttt{a || b}      \\
    \hline
    \multicolumn{3}{c}{\emph{Precedence: assignment}} \\
    \texttt{=}  & Simple assignment.   & \texttt{x = y}  \\
    \texttt{+=} & Add and assign.      & \texttt{x += y} \\
    \texttt{-=} & Subtract and assign. & \texttt{x -= y} \\
  \end{tabular}
\end{table}

\pending{}
Difference between logical and bitwise boolean operators.
Hat is not power or exponentiation, it is exclusive or.
Remind difference between equality and assignment.
Explain += and ++


\section{Statements}
\section{Operators}
\section{Flow Control}
\section{Functions}
\section{Structures}
\section{Classes}


%%% Local Variables:
%%% TeX-master: "main"
%%% eval: (adaptive-wrap-prefix-mode t)
%%% eval: (visual-line-mode t)
%%% eval: (nlinum-mode t)
%%% TeX-engine: luatex
%%% End: